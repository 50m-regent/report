\documentclass[titlepage]{jsarticle}
\usepackage[dvipdfmx]{graphicx}

\title{
  熱中症予防について \\
  \large ペットの熱中症予防
}

\date{}

\author{
  制御3年 \\
  32番 \\
  平田 蓮
}

\begin{document}
  \maketitle
  \begin{flushright}
    平田 蓮
  \end{flushright}
  \section{ペットの熱中症予防について}
    私はペットに犬を1匹飼っている。先月定期検診で動物病院に連れて行った時に、
    獣医から「ペットも熱中症になる」という話を聞いた。私は、ペットが熱中症になるという話を初めて聞き、
    驚きとともに自分の危機感の少なさを感じた。

    \begin{figure}[h]
      \begin{center}
        \includegraphics[width=10cm, angle=270]{koko.jpg}
        \caption{ペットのココ}
      \end{center}
    \end{figure}

    確かに、犬は汗をかかずに、舌で体温を調節しているので、体温調節が間に合わず、
    熱中症になる可能性は十分に高いように感じられる。

    今回、病院でこの話を聞いたあと、熱中症に関する課題が出たので、
    良い機会と思い、動物(特にペット)の熱中症について調べたいと思う。

    まずはじめに、そもそも熱中症とは何か。
    全日本病院協会\cite{zbk}によると熱中症とは、「気温が高い場所に長時間いることで、体温が調節できないレベルまで上昇してしまい、
    体調を崩す病気」である。
    これを踏まえると、人間であるないを関わらず、動物であれば熱中症になる危険性がある。

    次に、獣医から聞いた熱中症対策についてまとめる。

    \begin{itemize}
      \item 室内の換気に気をつける。
      \item 留守番をさせる時は、エアコンをつける。
      \item 散歩は、朝、気温が上昇する前と、夜、地面の温度が下がってから。
    \end{itemize}

    こうしてみてみると、人間の熱中症対策と何ら変わらないことがわかる。

    また、インターネットなどでもペットの熱中症に関する情報を収集した。
    ここでも、獣医の方に聞いたお話と大差はなかった。

    これらの対策のうち、上の二つはすでに実行していた。しかし三番目については、あまり意識したことがなかった。
    そもそも小型犬なので、あまり散歩はしないが、普段地面の温度を感じることはないので、
    昼間は地面が熱いことに気づかなかった。これからは散歩の時間も意識をしたいと思う。

    次に、もしペットが熱中症になってしまった場合の対応についてまとめたいと思う。
    インターネットなどで収集した情報を以下に記す。

    \begin{itemize}
      \item 涼しい場所に移動させる。
      \item 体を冷却させる。
      \item 水分と塩分を補給する。
    \end{itemize}

    そもそも熱中症は、体温が上昇しすぎた故に発症するものなので、
    体温を下げることに重きをおけば良いことがわかる。

    今回調査したことを参考に、安全に夏を乗り切りたいと思う。
  \section{授業の感想}
    前期に行ったソフトボールの授業はとても楽しかった。
    ソフトボールは、部活などに所属していない限り授業以外ではあまり行わない印象があり、
    実際今回初めて行った。
    貴重な機会を有効活用し、授業を楽しむことができたと思う。
    後期の授業もこの調子を続けたい。
  \begin{thebibliography}{99}
    \bibitem{} キッズネット 動物はあせをかくの https://kids.gakken.co.jp/kagaku/kagaku110/science0102/
    \bibitem{zbk} 全日本病院協会 熱中症について https://www.ajha.or.jp/guide/23.html
  \end{thebibliography}
\end{document}
