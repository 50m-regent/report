\documentclass[]{jsarticle}
\usepackage[dvipdfmx]{graphicx}
\usepackage{bm}
\usepackage{amsmath}
\usepackage{amssymb}
\usepackage{amsfonts}
\usepackage{comment}
\usepackage{listings}
\usepackage{cases}
\usepackage{siunitx}
\usepackage[hyphens]{url}
\lstset{
    basicstyle={\ttfamily},
    identifierstyle={\small},
    commentstyle={\smallitshape},
    keywordstyle={\small\bfseries},
    ndkeywordstyle={\small},
    stringstyle={\small\ttfamily},
    frame={tb},
    breaklines=true,
    columns=[l]{fullflexible},
    numbers=left,
    xrightmargin=0zw,
    xleftmargin=3zw,
    numberstyle={\scriptsize},
    stepnumber=1,
    numbersep=1zw,
    lineskip=-0.5ex,
    keepspaces=true,
    language=c
}
\renewcommand{\lstlistingname}{リスト}
\makeatletter
\newcommand{\figcaption}[1]{\def\@captype{figure}\caption{#1}}
\newcommand{\tblcaption}[1]{\def\@captype{table}\caption{#1}}
\makeatother

\title{受験報告書}
\date{}

\begin{document}
\maketitle
\section*{概要}
    \begin{description}
        \item[受験校] 横浜国立大学 理工学部 数物・電子情報系学科 情報工学教育プログラム
        \item[日程] 出願 5/14〜5/20、試験日 7/3
        \item[結果] 合格
    \end{description}
\section*{移動}
    前々日に東京の従兄弟の家に移動しました。
    他の大学の受験もあったので7/1から夏休み開始の7/22まで
    特別欠席をもらい一足早い夏休みを勝手に開催しようと企んでいましたが、
    案の定学生課に呼び出され、
    普通は前日から移動するので前々日は特別欠席にできないと言われ、
    見事に欠席がつきました。

    前日は一緒に横国を受ける友達と横浜に移動し、
    ホテルにチェックインした後下見にホテルから学校まで歩きました。
    最寄り駅まで徒歩15分、大学まで徒歩35分というなんとも言えない場所のホテルだったので、
    駅から電車に乗った後再び大学まで15分歩かないといけないことも考慮して当日も歩くつもりでした。
    しかし、梅雨時なのもあり、豪雨の中それなりの坂道と、非常に良い運動となりました。

    当日も朝から大雨となり、
    決して涼しくはない気温の中スーツに傘で35分歩き、
    大学につく頃には疲弊していました。
    バスを使うという選択肢もありましたが、
    大学からの案内には平日のバスの時刻しか書いていないにも関わらず
    受験日は土曜日という意味不明な状況を前に断念しました。
\section*{試験内容}
    数学、物理、論理回路、アルゴリズム、プログラミングと面接があります。
    \subsection*{数学}
        例年通り線形代数の問題と微分方程式が二問出題されました。
        微分方程式は特に語ることはありません。
        線形代数は対角化をするだけの典型問題でしたが、
        なぜか計算が合わずに少し手こずりました。

        どちらも過去問を解いておけば対策としては十分です。
    \subsection*{物理}
        例年通り力学の問題が出題されましたが、
        力積と力学的エネルギーの複合した、見慣れない問題で小問4つのうち2つしか解けませんでした。

        過去問を解いて傾向をつかみつつ、
        東京図書の「弱点克服 大学生の初等力学」という本で勉強するのがお勧めです。
    \subsection*{論理回路}
        基本的な内容が出題されました。
        稀にフリップフロップを使った順序回路の構成が出題されるので対策していましたが
        杞憂でした。

        2年のディジタル工学基礎と3年のディジタル論理回路の内容を復習しておけば十分です。
    \subsection*{アルゴリズム}
        アルゴリズムは例年アルゴリズムというよりはプログラムの読解問題が出題されます。
        2年の情報処理のテストを思い浮かべれば良いと思います。

        今年は掛け算と割り算をビット演算だけで実装しているプログラムでした。
        うまく構造が読めれば迷うところはなかったと思います。
    \subsection*{プログラミング}
        CとJavaの問題が1問ずつ出題されます。
        基本的にJavaはクラスやインターフェイスが理解できているかを問われるので、
        授業で学ぶレベルのCの知識があれば少し勉強するだけで十分です。

        Cは毎年何かのデータ構造が出題されます。
        今年は線形リストでした。

        過去問でプログラムを読み慣れておけば苦労しない内容です。
    \subsection*{面接}
        面接官は3人いましたが2人は一回も口を開きませんでした。
        面接で聞かれた内容を軽く書いておきます。
        \begin{itemize}
            \item 志望動機
            \item 卒研内容
            \item 試験の自己評価
            \item 二年次編入になる可能性もあるけど大丈夫か
            \item 併願校
        \end{itemize}
\section*{雑記}
    (少なくとも大学周辺の)横浜はみなさんが想像しているような夢溢れる都会ではありません。
    どこを歩いても上りか下り坂のひたすら山の街です。
    大学内は平坦で自然に溢れる良い大学ですが、正直そこの点では高専と変わりません。
    YouTuberのヨビノリたくみとゆきりぬの出身大学というのと、
    ネームバリューに釣られて受験しましたが、
    通い辛い大学だと思いました。

    全ての試験が一日で行われるのですが、
    面接の前に採点を終わらせておきたいのか、
    各試験間の待機時間が果てしなく長いです。
    それにも関わらず受験会場では一日を通して携帯電話は使用禁止なので
    本など暇を潰せるものを持っていきましょう。
    また、自販機が受験会場の建物内になかったのでわざわざ買いに行きたくない人は
    飲み物を一日分持ち込みましょう。

    服装ですが、面接があることもあり一人を除いてスーツでした。
    スーツを着る際は革靴を履くと思いますが、
    大学周辺で登山を要求されるため歩きやすい靴を履いて行って履き替えれば良いと思います。

    最後に、出願の際に担任の推薦書が必要です。
    事前にお願いしておきましょう。
\end{document}