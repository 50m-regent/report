\documentclass[]{jsarticle}
\usepackage[dvipdfmx]{graphicx}
\usepackage{bm}
\usepackage{amsmath}
\usepackage{amssymb}
\usepackage{amsfonts}
\usepackage{comment}
\usepackage{listings}
\usepackage{cases}
\usepackage{siunitx}
\usepackage[hyphens]{url}
\lstset{
    basicstyle={\ttfamily},
    identifierstyle={\small},
    commentstyle={\smallitshape},
    keywordstyle={\small\bfseries},
    ndkeywordstyle={\small},
    stringstyle={\small\ttfamily},
    frame={tb},
    breaklines=true,
    columns=[l]{fullflexible},
    numbers=left,
    xrightmargin=0zw,
    xleftmargin=3zw,
    numberstyle={\scriptsize},
    stepnumber=1,
    numbersep=1zw,
    lineskip=-0.5ex,
    keepspaces=true,
    language=c
}
\renewcommand{\lstlistingname}{リスト}
\makeatletter
\newcommand{\figcaption}[1]{\def\@captype{figure}\caption{#1}}
\newcommand{\tblcaption}[1]{\def\@captype{table}\caption{#1}}
\makeatother

\title{受験報告書}
\date{}

\begin{document}
\maketitle
\section*{概要}
    \begin{description}
        \item[受験校] 京都大学 工学部 情報学科 数理工学コース
        \item[日程] 出願 6/24〜7/5、試験日 8/31, 9/1
        \item[結果] 合格
    \end{description}
\section*{移動}
    前日に河原町のホテルに移動しました。
    街中にあるホテルにもかかわらず一泊3500円程度でした。

    7月に下見に来ていたので前日は下見に行きませんでした。
    当日は知り合いが車で送ってくれました。距離は5kmくらいです。
    バスが大量に走っているので周辺で移動に困ることはないと思います。

    大学の敷地内に会場案内などは一切なかったので迷子にならないように気をつけましょう。
\section*{試験内容}
    数学、物理、化学、面接があります。
    \subsection*{数学}
        微分方程式、重積分、線形代数、確率から4問出題されます。
        京都の微分方程式は完全微分形がよく出題されるのが特徴です。

        また、難しいというよりは、計算が多くめんどくさい問題も多い印象です。
        
        今年は例年より易化しており、点を落とした受験生は少なかったと思います。
    \subsection*{物理}
        力学と電磁気が出題されます。

        毎年どちらかは基本問題でもう片方は理論を問う問題になることが多いです。
        今年は力学が運動方程式から運動量に関する法則を導く問題で、
        電磁気は高専の授業で扱うレベルの電界、磁界に関する簡単な問題でした。
    \subsection*{化学}
        正直何もわかりませんでした。

        授業で化学がない分独学で頑張ろうと思いましたが、
        気づいたら受験直前になってしまい、
        過去問を数年分調べながら解いてみることしかできませんでした。
        
        炎色反応の問題と謎の計算問題以外解けていないと思います。
        後者の内容は何も理解していませんが、次元解析を行って頑張りました。

        分野は理論化学、有機化学です。
    \subsection*{面接}
        面接官は4人でした。
        物理の力学に関する口頭試問が少しありました。
        質問内容を書いておきます。
        \begin{itemize}
            \item 口頭試問
            \begin{itemize}
                \item バネによる単振動について
                \item 単振り子について (振幅が微小である場合とそうでない場合)
            \end{itemize}
            \item 志望動機
            \item 試験の自己評価
            \item 好きな科目や分野
            \item 熱力学について
            \item 卒研について
        \end{itemize}

        京都の面接はとても質問が鋭いです。
        自分が話したことについてどんどん掘り下げられるので
        自分が何を喋っているか常に気をつけましょう。

        全く対策できない不意打ちのような質問も飛んでくるので
        頑張ってその場で考えて言葉を捻り出しましょう。
\section*{雑記}
    7月に奈良高専の同じコースの志望者と仲良くなりました。
    このコースの志望者は2人だけで、
    毎年各コース2名くらい受かるので2人とも受かると思っていたら
    その人は落ちてしまいました。残念。

    出願の際にTOEFLのスコアを提出しないといけないので早めに受験しておきましょう。
    例年ボーダーは45から50点のようです。
    正直、今年の合格者をみるに半分くらいはTOEFLの点を見てる気もするのでTOEFLは頑張り得です。
\end{document}