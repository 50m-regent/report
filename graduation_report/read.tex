\documentclass[]{jsarticle}
\usepackage[dvipdfmx]{graphicx,color}

\begin{document}
\section*{1}
\section*{2}
    まず初めに、研究背景をお話しします。
    現在、スポーツ指導において、従来から行われてきた、
    指導者の経験に基づく感覚的な指導に比べ、
    科学的に解析した選手の動きに基づく定量的な指導が注目されています。

    これの実現例として、現在バレーボール業界で実用されているソフトウェアがあります。
\section*{3}
    定量的な指導を実現するソフトウェアとして「Data Volley」を紹介します。
    「Data Volley」は、イタリア発祥のバレーボール解析ソフトウェアです。
    アナリストと呼ばれる専門家が試合映像を見ながら、
    リアルタイムで選手のプレイをコマンドで入力していき、
    後からプレイの決定率などを解析できるというものです。

    しかし、これにはいくつかの欠点が挙げられます。
\section*{4}
    一つ目は、こちらに示してある例のように、入力コマンドが複雑であったり、
    専門的なプレイの判断が瞬間的に要求されるため、
    ソフトウェアを利用する学習コストが高いことです。
    二つ目は、選手の位置を記録する最小単位が1.5m四方と、
    あまり正確ではないことです。
    最後に、そもそも人が入力を行うため、人的入力ミスが起こりうる点が挙げられます。

    これらを踏まえて、本研究ではスポーツにおける選手の動きの解析のベースとして、
    次の条件を満たすシステムを目指しました。
\section*{5}
    まず、バレーボールの試合映像を解析します。

    次に、人的ミスが起きないようにできる限り人の手を介さずに解析を行います。

    次に、多くの人が使えるような扱いやすさを目指します。

    最後に、本研究では選手の動きの中でも特に位置に注目し、これを追跡します。

    これらを満たすシステムを開発するための方法を説明します。
\section*{6}
    まず映像内から選手を検出するのに、AlphaPoseを用います。
    AlphaPoseは、画像内の人物の位置・姿勢の検出アルゴリズムです。
    これは、一枚の画像を入力とするので、
    撮影した映像を各フレームごとに処理を行います。
\section*{7}
    AlphaPoseでは、画像に示されているように人物の姿勢が得られますが、
    本研究では、選手の床上の位置として、両足の中点を用います。

    次に、画像内の選手の位置から実世界における選手の位置を推定する方法を説明します。
\section*{8}
    選手の位置を推定するにあたって、カメラとコートの位置関係が必要です。
    撮影した映像内でコートがどの位置に写っているかを自動で検出することは
    困難であったため、本研究ではカメラを固定し、
    その際に映像のどこにコートの四隅が写っているは既知であるとして解析を行います。

    本研究では、手動で画像内からコートの四隅を指定できるソフトウェアを作成しました。
    先ほど、できる限り人の手を介さないといいましたが、
    本研究ではこの作業以外は全て自動で解析を行いました。
\section*{9}
    以上を用いて選手の実際の位置を推定するのに、射影変換を用います。
    左の画像は、バレーボールの試合の様子です。
    先述のAlphaPoseを用いて各選手の両足の中点を緑色のマーカーで示しています。
    これを、既知のコートの四隅の点を用いて射影変換を施すと、右の画像になります。
    これは、コートの四隅が実際のコートを上から見た形に一致するように変換が施されています。

    画像を見るとわかるように、各選手の位置を示す緑のマーカーもコートの変形に合わせて
    実際の選手の位置に移動していることがわかります。
    このことを用いて本研究では選手の位置を推定しました。

    これらのアルゴリズムを用いて得られた最終的な推定結果を示します。
\section*{10}
    左に示す映像は、Apple社のiPad Proを用いて撮影しました。

    この映像から、選手の位置を推定したものが右の映像です。
    人間の目でみると、正確にといえる程度に選手の位置が推定できていることがわかります。

    次に、推定した選手の位置の精度を評価しました。その方法を説明します。
\section*{11}
    精度を評価するにあたって、選手の推定位置と、同時刻の正確な位置が必要ですが、
    正確な位置を追跡するのは困難です。
    そのため、コート上に基準点を定め、選手がその上にいるときの推定位置の精度を評価しました。

    用いた基準点は図に示すとおりです。主に、コートの各ラインの交点を用いました。
\section*{12}
    左の画像に写っている選手は、先述の1番の基準点に立っている瞬間を切り取ったものです。
    このように、選手の実際の位置がわかる瞬間に注目して評価を行いました。

    また、カメラの位置による精度の変化をみるために複数の地点から撮影した映像を用いました。
\section*{13}
    図に示すように、コートを正面からの他に、
    コートの左右から斜めに映像を撮影しました。
    この際、カメラとコートの中心までの距離は20mに合わせました。

    精度の評価結果を示します。
\section*{14}
    表は、それぞれの角度から撮影した映像について、
    各基準点における推定点の誤差の平均を示しています。
    また、各値の横の括弧には映像から得られたサンプル数を示しています。

    表を見ると、各基準点での誤差は高々30cm程度に収まっていることがわかります。

    次に、カメラと基準点の距離と誤差の関係について示します。
\section*{15}
    図は、先ほど示した結果の各基準点について、カメラとその基準点の距離とそこでの平均誤差を
    示しています。
    図を見ると、実際に距離が増すほど誤差が増加しており、
    先ほどの予想が正しいことがわかります。

    また、全推定点の誤差の平均は0.21mと、
    1.5m四方が最小単位であるData volleyに比べて
    有位な位置の記録ができることがわかります。

    次に、正面から撮影した映像の基準点1に注目したときの
    推定点の分布を示します。
\section*{16}
    図は、上がカメラからみて奥、下がカメラから見て手前になるように向きを合わせた、
    推定点の分布です。赤い点が正確な基準点で、青が推定によって得られた選手の位置です。
    また、点線は、誤差の平均を半径とした円です。

    図を見ると、縦長の楕円状に分布していることがわかります。
    これは、用いた映像内でコートが縦長に引き伸ばされているため、
    左右の誤差よりも上下の誤差が大きく現れたためと考えられます。
\section*{17}
    本研究のまとめです。
    まず、バレーボールの試合の映像と、
    その中でどこにコートが写っているかという情報のみから
    選手の位置を推定しました。

    次に、その推定点の精度の評価を行いました。
    その結果、誤差の平均は0.21mとなり、
    1.5m四方で選手の位置を記録するData Volleyに対して
    有位であることがわかりました。
\section*{18}
    最後に、本研究の展望をお話しします。
    まず、AlphaPoseを用いて得られた選手の姿勢情報から、
    選手の三次元姿勢を推定し、それを用いることで
    バレーボールの試合の三次元再現が可能であると予想します。

    次に、本研究では選手の足が床についている前提で解析を行いましたが、
    実際のバレーボールの試合では選手はジャンプを行うため、
    空中にいる選手の位置推定について検討を行う必要があります。

    また、選手の位置だけでなく、ボールの位置も空中のため現時点では推定していませんが、
    これを推定することで、
    選手の動きだけではなく試合中の選手のプレイを解析することができ、
    より選手の指導に役立てられると考えています。
\end{document}
