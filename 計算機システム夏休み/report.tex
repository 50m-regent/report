\documentclass{jsarticle}

\title{書評レポート\\"How to Get a Job at Google"}
\author{EC3 32番 平田蓮}
\date{2019/9/7}

\begin{document}
\maketitle
\section{はじめに}
    今回、レポートを書くにあたって、New York Timesの記事である "How to Get a Job at Google" を読んだ。
    2014年の記事ということで、現在のGoogleとは違う点もあるかもしれないが、とても参考になったと思う。
    Part1、2を通してもそこまでの文量ではないので、より細かい点に注意を向けて読むことができたと思う。
\section{仕事を得るには}
    記事の著者であるThomas L. FriedmanはGoogleのLaszlo Bockにインタビューを行いこの記事を書いている。
    記事の中で、Bockが挙げたGoogleが雇用をするときに見るいくつかのポイントが紹介されている。
    以下は私が訳したものである。
    \begin{itemize}
        \item 汎用認知能力
        \item リーダーシップ
        \item 知的な謙虚さ
        \item 義務感と責任感
    \end{itemize}
    これらについての説明の中で、印象に残ったフレーズがいくつかある。

    中でも気になったのは、
    "Were you president of the chess club? Were you vice president of sales? How quickly did you get there? We don't care."
    という部分である。
    これは、上記の二つ目のリーダーシップの説明部分に書かれている。
    簡単に訳してみると、
    「今まで高い地位にいたのか? その地位にたどり着くのにどれだけ時間がかかったか? そんなことは気にしない。」
    とでもなるだろうか。
    この部分は今の日本人にとても響くのではないかと感じた。
    企業や学校に自分のアピールをするとき、真っ先に自分の経歴を挙げる人は多いと思う。
    Googleはその部分ではなく、突発的な任務に対してリーダーシップを持って取りかかれるかということを見るそうだ。
    これはその場ですぐできることではなく、長年の練習が必要なことである。
    改めてGoogleの着眼点に感銘を受けた。

\section{学校に行く必要はあるか}
    Part2では、主に今の学生に対するアドバイスが書かれていた。
    その中で「大学に行く必要があるのか」ということについて書かれていた。

    "The first and most important thing is to be explicit and willful in making the decisions about what you want to get out of this investment in your education."
    私はこの文を、
    「一番大事なことは、その投資から何を学び得たいのかをはっきりさせ、貪欲になることである。」
    と捉えた。
    Bockは、
    「学校に行くのは構わないが、ただそれが正しいことだと盲信するのはよくない」、
    また、
    「莫大な時間と金を投資した上で何を得られるのかをよく考えなければならない」
    とも述べている。

    私はこの意見にとても賛成である。
    義務教育を終え、特に高専という学校に通っている我々は、皆それなりに学びたいものがあって
    この進路を選んだはずである。

    私は専門的な知識を今の年齢のうちから学びたくてこの学校を選び、とても良い選択をしたと思っている。
    この学校に来たからあった出会いを自分の成長に活かしたい。

\begin{thebibliography}{99}
    \item Thomas L. Friedman, "How to Get a Job at Google", \\
        https://www.nytimes.com/2014/02/23/opinion/sunday/friedman-how-to-get-a-job-at-google.html
    \item Adam Bryant, "In Head-Hunting, Big Data May Not Be Such a Big Deal", \\
        https://www.nytimes.com/2013/06/20/business/in-head-hunting-big-data-may-not-be-such-a-big-deal.html
\end{thebibliography}
\end{document}
