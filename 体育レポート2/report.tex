\documentclass[]{jsarticle}
\usepackage[dvipdfmx]{graphicx}

\title{体育学年末レポート}
\author{Ec3 32 平田蓮}
\date{}

\begin{document}
\maketitle
日本語を母語とする人ならば、誰しもが敬語というものに馴染みがあるはずである。
頭が良いクラスメイトも、人気の俳優も、総理大臣も、もちろんオリンピック選手も。
人々から一目置かれている彼ら、彼女らも、また一目置いている私たちも
日常的に敬語を使うからである。私たち一般人が届かないと思っているような
彼らにも私たちと同じような一面がある。中学生の頃、同級生に2020年東京オリンピック
の代表候補とまでいわれていた体操選手がいた。しかし、今彼女が何をしているか私は知らないし、
おそらく私が彼女の情報を探して実際に東京オリンピック開催時にTV放送をみることもない。
私と、おそらく中学校にいた友人たちの間では、彼女は「オリンピック選手」ではなく、
他の人たちと何ら変わりない「友人」のひとりに過ぎなかったからだ。

現代のオリンピック選手は、いわゆる「スター」の一面を持っている。
昨年はじめに引退を発表したレスリングの吉田沙保里選手、
数々の平泳ぎの世界記録や日本記録を打ち立てている競泳の北島康介選手など、
彼らは数多くの人の注目を集め、憧れを持つ人も少なくない。
時として「スター」になって世界で活躍する彼らも、
オフシーズンは会社員であったり、いわゆる我々一般人と変わりないのである。
そんなオリンピック選手に注目しながら今までのオリンピックの歴史を振り返りつつ、
学年末レポートとする。

まず、現代のオリンピック(以下: 近代オリンピック)の起源であるオリンピア祭典競技、
いわゆる古代オリンピックについて触れる。
正確ではないが、古代オリンピックが始まったのは紀元前9世紀頃であると考えられている。
\cite{history}
近代オリンピックはスポーツの祭典であるが、
オリンピア祭典競技は神々を崇める宗教行事であった。
その頃のオリンピック選手は、文字通り「英雄」として讃えられた。
神々を崇める行事に参加した英雄で、実際に一般人とは一線を画す存在であった。
彼らは、時として一般人と同じである現在のオリンピック選手よりもさらに非一般的な神に近い存在
であったのだ。
しかし、この頃は女子禁制の行事で、選手は全員男子であった。
現在のように女子選手が登場したのは近代オリンピックになってからである。

近代オリンピックが始まったのは1896年である。
しかし、第一回のアテネ大会ではまだ男子のみの選手が参加した。
参加選手数は200人強と、2016年のリオデジャネイロオリンピックの11000人
には遠く及ばない人数である。
当時は、世界的な大会としてではなく、ヨーロッパの先進国同士が集まって開催をしたので、
参加国も少なく、規模はとても少ないものであった。
選手の知名度も現在とは打って変わり、
古代オリンピックとは真逆に、普通の一般人と変わる点は少なかった。

その後、近代オリンピックは徐々にその知名度を上げ、
今の知名度を誇るに至った。

\begin{thebibliography}{99}
    \bibitem{history} オリンピックの歴史 日本オリンピック委員会 \\
        https://www.joc.or.jp/sp/column/olympic/history/001.html
        (閲覧日: 2020/2/15)
\end{thebibliography}
\end{document}