\documentclass[uplatex]{jsarticle}

\title{書評レポート\\『AIの衝撃』、『人工知能は人間を超えるか』}
\author{EC3 32番 平田蓮}
\date{2019/5/11}

\begin{document}
\maketitle
\section{はじめに}
    今回書評レポートを書くにあたって、小林雅一著『AIの衝撃』と松尾豊著『人工知能は人間を超えるか』を読んだ。
    当初は『AIの衝撃』についてのみ書こうと考えていたが、先日、私の所属するプレラボのチームが
    松尾氏が主催する高専DCONに出場した際に興味を抱いたので、家にある著書を探し、『人工知能は人間を超えるか』を手に取った。

\section{AIに仕事が奪われる}
    『AIの衝撃』全体を通して、著者の「AIに携わる人間にならねばならない」という想いを感じた。
    特に、「AIに仕事が奪われる」という表現が印象に残っている。

    中でも著者の強い想いが描かれていたのは第3章、「日本の全産業がグーグルに支配される日」である。
    そのタイトルからもわかるように、最先端の人工知能研究をしているグーグルによって日本の産業が人工知能に支配されるという内容が書かれている。

    しかし著者は、まだ新しい人工知能という技術において、日本の遅れはまだ取り返せると述べている。
    そのためには、上でも述べたように、人工知能に携わる人間を増やさねばならないという。
    まだ人工知能という分野には世界で50人程度しか専門家がいないという説まであるなか、
    今から新しい人材を育成することで日本が世界に追いつくことは十分可能なのかもしれない。

    著者は人材の育成の他にも、優秀な人材が海外に逃げてしまう問題についても述べている。
    近年、優秀な大学の卒業生などは、海外の企業や、その日本法人(例によってグーグルジャパンなど)に就職してしまうことが多いという。
    著者はこれを受け、これは日本の文化や、懐古的なシステムなどに問題があり、これから変えていく必要があると述べている。

    確かに、この点については間違いないと筆者も読んでいて感じた。しかし、一日本の学生として、
    筆者は根本的な教育システムにも問題があるのではないかと感じる。最近は中高生のうちから自主的にプログラミングを始める人が増えているが、
    近年では中高からではなく、小学生のうちからプログラミング教育をしようとする動きもあるようだ。
    筆者はこの部分において疑問を抱いている。確かに若いうちからプログラミングの知識を学ぶことは重要かもしれない。
    しかし、身近な人をみていても、プログラミング以前の基礎的な数学(算数)などの教育を重要視しなければならないのではないかと感じる。

    たまに話題になるように、小学校の教師の中には、本質的ではない丸つけをするような人などもいる。
    全ての教師があのような訳ではないが、新しい教育をする以前の問題が残っているように感じる。
    そもそも、教える側の人間が新しく教えることを正しく理解しているのであろうか。\\

    さらに著者は、「人工知能に\textquotedblleft 支配\textquotedblright され、仕事を失わないためにも人工知能を作る側の人間にならなければならない」と述べていたが、
    筆者には疑問に思うことがある。将来人工知能がさらに高いレベルまで成長し、今人間がしている仕事を代わりにする時代が来ても、
    果たしてそれは憂うべき事態なのか。生産、販売の過程に人間の手が入らなくなったことによって人件費などが浮き、
    結果として人間はあまり働かなくても生活していけるようになるのではないか。この疑問は長年持っていたが、
    改めて再認識したのでいつか調べる課題としておく。

    また『人工知能は人間を超えるか』でも、著者の松尾氏が人工知能の産業に対する影響に関して述べている。
    この章の内容に関して、次の節で述べる。

\section{人工知能が産業を潰す}
    前節でも述べたように、松尾氏の『人工知能は人間を超えるか』の終章「変わりゆく世界」にても、『AIの衝撃』の第3章と同じように人工知能の産業に対する影響について書かれていた。
    その章では、今後開発が進むであろう人工知能の分野が時代別に表に示されていた。しかし、開発されたとしても、実際に社会で使われるまでは長い時間がかかる場合もあるという。

    また、その後前述の表に基づいた上で、これもまた時代別に、「無くなる職業と残る職業」についてまとめられている。
    小林氏の『AIの衝撃』では日本の産業に特化して述べられていたが、こちらでは、日本に限らず、世界の様々な産業について述べられている。
    さらに、今存在してる職業に限らず、今後「人工知能が生み出す可能性のある職業」についても述べられていた。
    この章を通じて、著者の、「人工知能は必ずしも\textquotedblleft 悪\textquotedblright ではない」という強い想いを感じた。

    松尾氏は章の最後にて、「日本における人工知能発展の課題」と題して、『AIの衝撃』でも述べられていた、日本の現状について述べている。
    そこでは日本の今の課題として大きく五つ取り上げている。しかし、小林氏とは違い、「日本には古くから優秀な人工知能の人材がいる」とまとめている。

\section{どこまでAIをつくるか}
    『AIの衝撃』を読んでいて、終盤にとても印象に残っている部分がある。それは、著者である小林氏が
    「そもそも人間を超えるものを人間がわざわざつくるか」と述べているところである。

    人工知能の研究が始められた当初は、人工知能というのは、ある一部の分野に特化されたものであったが、
    今の時代はそれも変わり、「強いAI」と呼ばれる、様々な分野に能力を活かせる汎用的な人工知能が生まれてきている。
    これから先、これらの汎用的な能力がさらに強化され、ついに人間を超えようとするとき、人間はその人工知能を果たして作るのか。
    と、著者は述べている。

    私はこの意見を初めて知った。今まで考えもしなかったことで、とても印象に残った。
    しかし、私は人間の意志が関係ないのではないかと考える。完全に汎用的な人工知能が人間によって生み出される前に、
    「人工知能が人工知能を開発する」ことができるのではないか。実際に、人間が人工知能を開発する際に決めるパラメータなどを決める
    人工知能を作ってみたこともある。

    このことは、『人工知能は人間を超えるか』にても綴られている。自分自身よりも強い人工知能を作れる人工知能が生まれた瞬間、
    人工知能が無限に生産される時代がくるという、「技術的特異点(シンギュラリティ)」と呼ばれる概念である。人工知能の専門分野を絞れば、
    今の時代、すでに人間を超えている人工知能は多数ある。シンギュラリティがくるのもそう遠くないのかもしれない。

\section{おわりに}
    今回読んだ二冊を通して様々な知見を得られた。特に松尾氏の『人工知能は人間を超えるか』では、人工知能の技術的な面についても細かく書かれていた。
    人工知能に興味がある工学徒としては、今までの知識をさらに広げられる大変興味深い内容だった。今後も人工知能分野のみにとらわれず、
    様々な分野の本を読んで、知見を身につけていきたい。

    また、授業内容に関して、つい先月受けた基本情報技術者試験に出題される内容に似ていると感じている。受験前の勉強ではあまり深く理解をすることができなかったので、
    これからの授業を通してさらに理解を深めたいと思う。また、昨年度のディジタル工学基礎や、今年度のディジタル論理回路などの科目とも関連する内容もあると思うので、
    それらと絡めて、知識を活用できるように学習していきたい。

\end{document}
