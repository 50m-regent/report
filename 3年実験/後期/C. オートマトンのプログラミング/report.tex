\documentclass[titlepage]{jsarticle}
\usepackage[dvipdfmx]{graphicx}
\usepackage{listings}
\usepackage{h31ec-exp}
\lstset{
  basicstyle={\ttfamily},
  identifierstyle={\small},
  commentstyle={\smallitshape},
  keywordstyle={\small\bfseries},
  ndkeywordstyle={\small},
  stringstyle={\small\ttfamily},
  frame={tb},
  breaklines=true,
  columns=[l]{fullflexible},
  numbers=left,
  xrightmargin=0zw,
  xleftmargin=3zw,
  numberstyle={\scriptsize},
  stepnumber=1,
  numbersep=1zw,
  lineskip=-0.5ex
}
\renewcommand{\lstlistingname}{ソースコード}
\makeatletter
\newcommand{\figcaption}[1]{\def\@captype{figure}\caption{#1}}
\newcommand{\tblcaption}[1]{\def\@captype{table}\caption{#1}}
\makeatother
\title{オートマトンのプログラミング}
\grade{3年32番}
\author{平田 蓮}
\team{第4班}
\date{2019年10月29日}
\expdate{2019年10月7日, 10月21日, 10月28日}
\coauthor{
    4番  石橋那起 \\
    8番  小林歩夢 \\
    12番  小室弦太 \\
    15番  佐藤貴幸 \\
    20番  関晋一朗 \\
    24番 高橋祐己哉 \\
    28番 外川諒太郎 \\
    36番  本多充稔
}

\begin{document}
\maketitle
\section{目的}
    本実験では, 擬似自動販売機回路のプログラムを作成し, プログラミングを通してオートマトンの
    考え方を理解する.

\section{有限オートマトン(Finite Automaton: FA)}
    オートマトンとは自動機械という意味であるが、工学で用いられる場合は、離散的な入力及び出力
    を持つ機械のモデルのことであり, 状態とその遷移という考え方で捉える.
    ある装置の動作を実現することを考えた場合に, 入出力をまず考えるが, それだけでは動作を実現することは
    できない. 出力を決定する要素として内部状態という考えが必要である.

    装置の取り得る内部状態の数が有限個の場合, その装置を有限オートマトンといい, その動作は次の5個の
    集合と関数で記述できる.

    \paragraph{FAに必要な集合と関数}
        上で述べた集合と関数を示す.

        \begin{itemize}
            \item $X$: 入力集合
            \item $Q$: 状態集合
            \item $Z$: 出力集合
            \item $\sigma$: 状態遷移関数 $\sigma(X, Q) \rightarrow Q$
            \item $\omega$: 出力関数 $\omega(X, Q) \rightarrow Z または \ \omega(Q) \rightarrow Z$
        \end{itemize}

    \subsection{状態遷移図}
        FAの動作を図で表すには状態遷移図を用いると良い.

        例として10円硬貨だけが使える30円切手自動販売機を考える.
        Cancelボタンを押すと払い戻しとする.

        \begin{itemize}
            \item $X$: \{10[円], Cancel\}
            \item $Q$: \{0[円], 10[円], 20[円]\} (初期状態: 0[円])
            \item $Z$: \{1[枚], 10[円], 20[円]\}
        \end{itemize}

        このFAの状態遷移図を図\ref{fig:例状態遷移図}に示す.
        以下, 全ての状態遷移図で入出力を入力/出力のように示す.

        \begin{figure}[ht]
            \centering
            \includegraphics[width=8cm]{images/rei.pdf}
            \caption{30円切手自動販売機}
            \label{fig:例状態遷移図}
        \end{figure}

\section{練習問題}
    実験テキストの練習問題の状態遷移図を示す.

    (1) 10円硬貨だけが使用できる40円切手自動販売機. Cancelを押すと払い戻し.

    \begin{figure}[ht]
        \centering
        \includegraphics[width=8cm]{images/40.pdf}
        \caption{40円切手自動販売機}
        \label{fig:40状態遷移図}
    \end{figure}

    (2) 10円硬貨と50円硬貨だけ使える30円切手自動販売機. Cancelを押すと払い戻し.
    (お釣りはCancelを押さないと出てこない.)

    \begin{figure}[ht]
        \centering
        \includegraphics[width=8cm]{images/30.pdf}
        \caption{30円切手自動販売機}
        \label{fig:30状態遷移図}
    \end{figure}

    (3) 10円, 50円, 100円硬貨が使える20円切手自動販売機. Cancelを押すと払い戻し.
    (お釣りはCancelを押さないと出てこない.)

    \begin{figure}[ht]
        \centering
        \includegraphics[width=8cm]{images/20.pdf}
        \caption{20円切手自動販売機}
        \label{fig:20状態遷移図}
    \end{figure}

\section{仮想自動販売機作成実習}
    今回は, 10円と50円が使える20円切手自販機を実装してみた.
    10円入ってるときに50円を入れると60円になり切手が3枚出力されるので,
    3枚目は100円のランプを使うこととする.
    また, Cancelを押すと払い戻しをする.

    図\ref{fig:本番状態遷移図}に状態遷移図を示す.

    \begin{figure}[ht]
        \centering
        \includegraphics[width=8cm]{images/honban.pdf}
        \caption{20円切手自動販売機状態遷移図}
        \label{fig:本番状態遷移図}
    \end{figure}

    \subsection{ソースコード}
        今回変更を加えた部分のソースコードを以下に示す. \\

        \begin{lstlisting}[caption=piclib.c, label=piclib]
void  StampOut(int num) {
    if (num > 2) led_on(LED100);
    if (num > 1) led_on(STAMP2);
    if (num > 0) led_on(STAMP1);
}
        \end{lstlisting}

        \begin{lstlisting}[caption=piclib.h extern.h]
void StampOut(int num);		/* スタンプ表示		*/
        \end{lstlisting}

        \begin{lstlisting}[caption=stamp.c 遷移関数, label=transition]
int Transition (char it, Status* st, int* a, int* b) {
    *a = *b = 0;	/* コインA,Bの出力枚数を0にしておく */
    if (it == Exit) *st = stExit;
    if (it == Cancel) {
        *a = *st;
        *st = stEmpty;
    }
    if (it == CoinA) {
        ++*st;
        *st = *st % 2;
        if (!*st) {
            return 1;
        } else {
            return 0;
        }
    }
    if (it == CoinB) {
        *st += 5;
        *st = *st % 2;
        if (*st) {
            return 2;
        } else {
            return 3;
        }
    }
    return 0;
}
        \end{lstlisting}

        \begin{lstlisting}[caption=stamp.c メインループ, label=main]
do {
    it = sw_read();
    if (it == 0) continue;
    outfg = outcoin = 0;
    s = Transition(it, &st, &a, &b);
    DispStatus(status_bit[st]);
    if (a | b) outfg = 1;
    StampOut(s);
    if (a) DispCoinA(a);
    if (b) DispCoinB(b);
    while(it = sw_read()){	/*  スイッチ入力監視	*/
        if (it == Exit) break;
        timer(30);
    }
} while (st != stExit);
        \end{lstlisting}

        まず, ソースコード\ref{piclib}にあるように, 切手の出力をする関数を複数枚出力ができるように変更した.
        StanpOut関数は引数で切手の枚数を受け取り, その値の個数だけLEDを光らせる.
        また, それに伴いpiclib.hとextern.hにあるStampout関数のプロトタイプ宣言を変更した.

        ソースコード\ref{transition}には, 変更後の遷移関数を示した.
        遷移関数Transitionは受け付ける入力を2種類に増やし,
        それぞれの入力に対し適切な切手枚数を返すように変更した.

        最後に, これらの変更に伴い, ソースコード\ref{main}のようにstamp.cのメインループ内を変更した.
    
\section{調査課題「ワンチップマイコンについて調査せよ」}
    コンピュータを形成するのに必要な要素は入出力装置, 記憶装置, 処理装置などである.
    これらを一つのICにまとめたものがワンチップマイコンである.

    ワンチップマイコンの特徴として, 以下のものが挙げられる.

    \begin{itemize}
        \item 小さい
        \item 安い
        \item 多種類
        \item プログラムの書き換えが可能
    \end{itemize}

    現在主流のワンチップマイコンは, Microchip社製のPICシリーズと,
    現在はMicrochip社に買収されたAtmel社製のAVRシリーズがある.
    今回の実験ではPICシリーズを使用した.

    AVRにはPICと比べて以下のような特徴がある.

    \begin{itemize}
        \item 高速
        \item 操作が簡単
        \item 知名度が低く, 情報が少ない
    \end{itemize}

    それぞれに特徴があり, 用途によって使い分けることが必要である.

\section{考察}
    今回は最大で入力集合の大きさが4の自販機を練習問題で考えたが,
    実際の自販機を考えると, 入力集合の大きさはより大きくなる.

    そのため, 実装するには状態遷移図, ソフトウェア共に工夫が必要になってくると考えられる.

    今回はソフトウェア面での工夫を考えてみた.
    今回の実験で使用したコードでは, 一つの遷移関数Transitionの中で状態の遷移を行なったが, 
    構造体やクラスを使って硬貨や自販機の処理との関係性をシミュレーションをすることで
    簡潔なコードで多種の入力がある自販機を実装することができると考えた.
    この方法をとることで出力種類の増加にも対応させることができる.

\section{感想}
    今回の実験では, 前期のディジタル論理回路の授業内容が活かせた.
    また, 今まで培ってきたプログラミング能力を駆使して比較的早く課題を終わらすことができた.
    発展課題には挑戦しなかったので, 今後機会があったら調べてみたい.

\begin{thebibliography}{99}
    \bibitem{Electro} Electro-ワンチップマイコン http://laboratory.sub.jp/ele/13.html/
    \bibitem{nmri} 第1章 ワンチップマイクロチップコンピュータとは \\
        https://www.nmri.go.jp/oldpages/eng/khirata/mcon/ch01.html/
    \bibitem{design spark} さらなる可能性を秘めたPIC, AVRマイコン \\
        https://www.rs-online.com/designspark/unlimited-possibilities-of-pic-and-avr-microcontrollers-jp/
\end{thebibliography}

\end{document}
