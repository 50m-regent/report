\documentclass[titlepage]{jsarticle}
\usepackage[dvipdfmx]{graphicx}
\usepackage{listings}
\usepackage{h31ec-exp}
\lstset{
  basicstyle={\ttfamily},
  identifierstyle={\small},
  commentstyle={\smallitshape},
  keywordstyle={\small\bfseries},
  ndkeywordstyle={\small},
  stringstyle={\small\ttfamily},
  frame={tb},
  breaklines=true,
  columns=[l]{fullflexible},
  numbers=left,
  xrightmargin=0zw,
  xleftmargin=3zw,
  numberstyle={\scriptsize},
  stepnumber=1,
  numbersep=1zw,
  lineskip=-0.5ex
}
\renewcommand{\lstlistingname}{ソースコード}
\makeatletter
\newcommand{\figcaption}[1]{\def\@captype{figure}\caption{#1}}
\newcommand{\tblcaption}[1]{\def\@captype{table}\caption{#1}}
\makeatother
\title{IoTシステム開発の基礎}
\grade{3年32番}
\author{平田 蓮}
\team{第4班}
\date{2019年11月19日}
\expdate{2019年11月6日, 11月11日, 11月18日}
\coauthor{
    4番  石橋那起 \\
    8番  小林歩夢 \\
    12番  小室弦太 \\
    15番  佐藤貴幸 \\
    20番  関晋一朗 \\
    24番 高橋祐己哉 \\
    28番 外川諒太郎 \\
    36番  本多充稔
}

\begin{document}
\maketitle
\section{目的}
    本実験では, Raspberry Piを用いてLinux環境を構築し, その手順とともにIoT(Internet of Things)
    システム開発の基礎を習得する.

\section{実験手順}
    \subsection{Linux環境の構築}
        \subsubsection{Raspbianのインストール}
            Raspberry Piでは内蔵されたストレージからではなく,
            microSDカードからOSを読み込んで起動する.
            そのため, Raspberry Piを使用するためにOSが入ったmicroSDカードを用意する
            必要がある. 今回はRaspberry Pi公式サポートOSであるRaspbianを使用する.

            以下に大まかな手順を示す.

            \begin{enumerate}
                \item OSファイルの準備
                \item Raspberry Piの起動, OSのインストール
                \item Raspbianの初期設定
            \end{enumerate}

            \paragraph{OSファイルの準備}
                必要なOSファイルは, 校内サーバーからダウンロードをして使用する.

                \begin{enumerate}
                    \item http://ftp.nagaoka-ct.ac.jp/pub/linux/raspbian
                        にアクセスしてNOOBS\_v3\_2\_0.zipダウンロードする
                        (このとき, デスクトップに保存しないと容量が足りないので気を付ける).
                    \item microSDカードをPCに接続し, 解凍したファイルの中身を全てコピーする.
                    \item デスクトップにあるzipファイルを削除する.
                \end{enumerate}

            \paragraph{Raspberry Piの起動, OSのインストール}
                \begin{enumerate}
                    \item もう一つのモニターからPCと周辺機器に接続しているケーブル類を外し,
                        今回の実験で使うものを代わりに接続する.
                    \item Raspberry PiにmicroSDカード, HDMIケーブル, 周辺機器のケーブルを
                        接続する.
                    \item Raspberry Piに電源ケーブルを接続し, 起動する.
                    \item 画面に出る指示に従いインストールを進める.
                \end{enumerate}

            \paragraph{Raspbianの初期設定}
                インストールが完了してRaspberry Piが起動したらLANケーブルをRaspberry Pi
                に接続し, 初期設定を進める.

                \begin{enumerate}
                    \item プロキシの設定
                    
                        Raspberry Piでターミナルを開き, 以下の内容を/etc/apt/apt.confに
                        書き込む.
                            
                    \verb|sudo vi /etc/apt/apt.conf|

                        \begin{lstlisting}[caption=apt.conf]
Acquire::http::proxy "http://st-proxy.st.nagaoka-ct.ac.jp:8080/";
Acquire::https::proxy "https://st-proxy.st.nagaoka-ct.ac.jp:8080/";
Acquire::ftp::proxy "ftp://st-proxy.st.nagaoka-ct.ac.jp:8080/";
                        \end{lstlisting}
                    \item あああ
                \end{enumerate}

    \subsection{IoTシステム作成}

    \subsection{温湿度, 気圧情報ページ作成}

\section{CGI作成課題} \label{kadai}


\section{考察}
    今回はセンサーから情報を読み取る際にGitHubにあったサンプルコードを使用した.
    そのため, html出力をする際にサンプルコード内のprint文を改変する必要があった.

    今回\ref{kadai}節ではやらなかったが, そもそもサンプルコードの関数から湿温度, 気圧情報をそれぞれ
    返り値として取得することで, cgiファイル内のPythonスクリプトでオリジナルの操作を施しやすくなると考えた.

    使いやすいライブラリを作成することも大切であると改めて実感した.

\section{感想}
    今回のテーマは実験を行うのが初めてということで, 資料に間違いなどが見受けられた.
    しかし, エラーに直面しても, 自分で調べたり, 今まで学習してきた知識を活用することで
    修正することができた. この力は今後とても大切になってくると感じた.
    これからも学習を怠らないようにしたい.

\end{document}
