\documentclass[titlepage]{jsarticle}
\usepackage[dvipdfmx]{graphicx}
\usepackage{listings}
\usepackage{h31ec-exp}
\usepackage{biblatex}
\lstset{
  basicstyle={\ttfamily},
  identifierstyle={\small},
  commentstyle={\smallitshape},
  keywordstyle={\small\bfseries},
  ndkeywordstyle={\small},
  stringstyle={\small\ttfamily},
  frame={tb},
  breaklines=true,
  columns=[l]{fullflexible},
  numbers=left,
  xrightmargin=0zw,
  xleftmargin=3zw,
  numberstyle={\scriptsize},
  stepnumber=1,
  numbersep=1zw,
  lineskip=-0.5ex
}
\renewcommand{\lstlistingname}{ソースコード}
\makeatletter
\newcommand{\figcaption}[1]{\def\@captype{figure}\caption{#1}}
\newcommand{\tblcaption}[1]{\def\@captype{table}\caption{#1}}
\makeatother
\title{シーケンサによる自動制御}
\grade{3年32番}
\author{平田 蓮}
\team{第4班}
\date{2019年12月17日}
\expdate{2019年12月23日, 1月6日, 1月20日}
\coauthor{}

\begin{document}
\maketitle
\section{目的}
  プログラマブルコントローラ(シーケンサ)による自動制御法(リレーラダー方式, ステップラダー方式)を学び, 課題実験のシステムの設計, 確認実習を行うことで理解を深める.
\section{クイズの解答表示システムの設計}
  次節に述べる仕様を満たす回路を作成する.
  \subsection{制御仕様}
    \begin{itemize}
      \item 司会者の出題するクイズに対して, もっとも早くボタンを押したデスクのランプを点灯させる.
        点灯後は司会者が押しボタン$PB_4$を押すまで点灯している.
        ただし, 子供チームの押しボタン$PB_{11}$と$PB_{12}$はどちらも押してもランプ$L_1$を点灯させることができるよう,
        有利になっている. また, 博士チームの押しボタン$PB_{31}$と$PB_{32}$は両方とも押さなければランプ$L_3$は
        点灯しないよう, 不利になっている.
      \item 司会者がスイッチSWをONにしたときに, 10秒以内に回答者のランプがついた場合, 電磁石SOLが働いてくす玉が
        割れるようなラッキーチャンスとなっている.
        割れたくす玉はラッキーチャンスが終わった後もその状態を保持し, 押しボタン$PB_4$を押すともとに戻る.
    \end{itemize}
  表\ref{tab:taiou}に上で示したボタン等とシーケンサのゲート番号との対応表を示す.
  \begin{table}[h]
    \caption{入出力接続対応表}
    \label{tab:taiou}
    \centering
    \begin{tabular}{c|c}
      記号 & シーケンサ \\ \hline \hline
    \end{tabular}
  \end{table}
\section{押しボタン式横断歩道の設計}
\section{課題}
\section{感想}
\section*{参考文献}
  \begin{enumerate}
    \item 令和元年度電子制御工学実験 $\cdot$ 3年後期テキスト
  \end{enumerate}
\end{document}
