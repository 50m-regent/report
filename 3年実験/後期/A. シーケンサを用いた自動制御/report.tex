\documentclass[titlepage]{jsarticle}
\usepackage[dvipdfmx]{graphicx}
\usepackage{listings}
\usepackage{h31ec-exp}
\usepackage{biblatex}
\lstset{
  basicstyle={\ttfamily},
  identifierstyle={\small},
  commentstyle={\smallitshape},
  keywordstyle={\small\bfseries},
  ndkeywordstyle={\small},
  stringstyle={\small\ttfamily},
  frame={tb},
  breaklines=true,
  columns=[l]{fullflexible},
  numbers=left,
  xrightmargin=0zw,
  xleftmargin=3zw,
  numberstyle={\scriptsize},
  stepnumber=1,
  numbersep=1zw,
  lineskip=-0.5ex
}
\renewcommand{\lstlistingname}{ソースコード}
\makeatletter
\newcommand{\figcaption}[1]{\def\@captype{figure}\caption{#1}}
\newcommand{\tblcaption}[1]{\def\@captype{table}\caption{#1}}
\makeatother
\title{シーケンサによる自動制御}
\grade{3年32番}
\author{平田 蓮}
\team{第4班}
\date{2019年12月17日}
\expdate{2019年12月23日, 1月6日, 1月20日}
\coauthor{}

\begin{document}
\maketitle
\section{目的}
  プログラマブルコントローラ(シーケンサ)による自動制御法(リレーラダー方式, ステップラダー方式)を学び, 課題実験のシステムの設計, 確認実習を行うことで理解を深める.
\section{クイズの解答表示システムの設計}
  次節に述べる仕様を満たすプログラムを作成する.
  \subsection{制御仕様}
    \begin{itemize}
      \item 司会者の出題するクイズに対して, もっとも早くボタンを押したデスクのランプを点灯させる.
        点灯後は司会者が押しボタン$PB_4$を押すまで点灯している.
        ただし, 子供チームの押しボタン$PB_{11}$と$PB_{12}$はどちらも押してもランプ$L_1$を点灯させることができるよう,
        有利になっている. また, 博士チームの押しボタン$PB_{31}$と$PB_{32}$は両方とも押さなければランプ$L_3$は
        点灯しないよう, 不利になっている.
      \item 司会者がスイッチSWをONにしたときに, 10秒以内に回答者のランプがついた場合, 電磁石SOLが働いてくす玉が
        割れるようなラッキーチャンスとなる.
        割れたくす玉はラッキーチャンスが終わった後もその状態を保持し, 押しボタン$PB_4$を押すともとに戻る.
    \end{itemize}
  \subsection{設計}
    表\ref{tab:taiou}に上で示したボタン等とシーケンサのゲート番号との対応表,
    図\ref{fig:quiz_lad}, 表\ref{tab:quiz_code}に設計したリレーラダー図, また, それのコーディングを示す.
    \begin{table}[h]
      \caption{入出力対応表}
      \label{tab:taiou}
      \centering
      \begin{tabular}{c|c|c||c|c|c}
        \hline
        記号 &       名前 &             シーケンサ & 記号 &   名前 &            シーケンサ \\ \hline \hline
        $PB_{11}$ & 子供チームのボタン1 & X400 &     $L_1$ & 子供チームのランプ & Y431 \\
        $PB_{12}$ & 子供チームのボタン2 & X401 &     $L_2$ & 学生のランプ      & Y432 \\
        $PB_2$ &    学生のボタン &       X402 &     $L_3$ & 博士チームのランプ & Y433 \\
        $PB_{31}$ & 博士チームのボタン1 & X403 &     SW &    司会者用スイッチ   & X406 \\
        $PB_{32}$ & 博士チームのボタン2 & X404 &     SOL &   くす玉の電磁石    & Y434 \\
        $PB_4$ &    司会者用ボタン &     X405 & & & \\ \hline
      \end{tabular}
    \end{table}
    \begin{figure}[h]
      \centering
      % \includegraphics[]{}
      \caption{リレーラダー図}
      \label{fig:quiz_lad}
    \end{figure}
    \begin{table}[h]
      \caption{コーディング}
      \label{tab:quiz_code}
      \centering
      \begin{tabular}{r|lr||r|lr||r|lr||r|lr||r|lr}
        0 & LDI & 431 & 10 & OUT & 431 & 20 & OUT & 101 & 30 & ORB & &     40 & OUT & 450 \\
        1 & ANI & 432 & 11 & LD & 402 &  21 & LD & 102 &  31 & OUT & 433 & 41 & K & 10 \\
        2 & ANI & 433 & 12 & AND & 100 & 22 & ANI & 405 & 32 & LD & 406 &  42 & LD & 103 \\
        3 & OUT & 100 & 13 & LD & 432 &  23 & OR & 404 &  33 & AND & 100 & 43 & ANI & 450 \\
        4 & LD & 400 &  14 & ANI & 405 & 24 & OUT & 102 & 34 & LD & 103 &  44 & ANI & 100 \\
        5 & OR & 401 &  15 & ORB & &     25 & LD & 101 &  35 & ANI & 405 & 45 & LD & 434 \\
        6 & AND & 100 & 16 & OUT & 432 & 26 & AND & 102 & 36 & ANI & 450 & 46 & ANI & 405 \\
        7 & LD & 431 &  17 & LD & 101 &  27 & AND & 100 & 37 & ORB & &     47 & ORB & \\
        8 & ANI & 405 & 18 & ANI & 405 & 28 & LD & 433 &  38 & OUT & 103 & 48 & OUT & 434 \\
        9 & ORB & &     19 & OR & 403 &  29 & ANI & 405 & 39 & LD & 103 &  49 & END & \\
      \end{tabular}
    \end{table}
\section{押しボタン式横断歩道の設計}
  今回の実験を通して新しくステップラダー方式を学ぶ.
  ステップラダー図はリレーラダー方式と違い,
  状態遷移図に基づいてプログラムを作成する.
  この節では, 以下の制御仕様を満たすようにステップラダー方式を使ってプログラムを作成する.
  \subsection{制御仕様}
    \begin{itemize}
      \item 横断ボタンX400またはX401が押されると, 図\ref{fig:sig_pat}のパターンで信号灯が
        切り替わる. 一連の動作中に押しボタンを押しても無効とする.
      \item 設計には並進分岐のステップラダーを使用し, 点滅にはカウンタを使用する.
        使用するタイマーでは図の時間のみ使用する.
    \end{itemize}
    \begin{figure}[h]
      \centering
      % \includegraphics[]{}
      \caption{信号の動作パターン}
      \label{fig:sig_pat}
    \end{figure}
  \subsection{設計}
    まず, 入出力対応表を示す.
    \begin{table}[h]
      \caption{入出力対応表}
      \centering
      \begin{tabular}{c|c||c|c}
        \hline
        名前 &       シーケンサ & 名前 &         シーケンサ \\ \hline \hline
        押しボタン1 & X400 &     車用青信号 &    Y432 \\
        押しボタン2 & X401 &     歩行者用赤信号 & Y433 \\
        車用赤信号 &  Y430 &     歩行者用青信号 & Y434 \\
        車用黄信号 &  Y431 & & \\ \hline
      \end{tabular}
    \end{table}

    次に, 状態遷移図, ステップラダー図, コーディングを示す.
    \begin{figure}[h]
      \centering
      % includegraphics[]{}
      \caption{状態遷移図}
    \end{figure}
    \begin{figure}[h]
      \centering
      % includegraphics[]{}
      \caption{ステップラダー図}
    \end{figure}
    \begin{table}[h]
      \caption{コーディング}
      \centering
      \begin{tabular}{r|lr||r|lr||r|lr||r|lr||r|lr}
        0 &  LD &  71 &  15 & STL & 601 & 30 & K &   5 &   45 & S &   607 & 60 & STL & 603 \\
        1 &  S &   600 & 16 & OUT & 432 & 31 & STL & 604 & 46 & STL & 607 & 61 & STL & 610 \\
        2 &  OUT & 671 & 17 & OUT & 450 & 32 & OUT & 433 & 47 & OUT & 434 & 62 & LD &  456 \\
        3 &  K &   601 & 18 & K &   30 &  33 & LD &  452 & 48 & OUT & 455 & 63 & S &   600 \\
        4 &  OUT & 672 & 19 & LD &  450 & 34 & S &   605 & 49 & K &   0.5 & 64 & RET & \\
        5 &  K &   610 & 20 & S &   602 & 35 & STL & 605 & 50 & LD &  455 & 65 & LD &  71 \\
        6 &  OUT & 670 & 21 & STL & 602 & 36 & OUT & 434 & 51 & AND & 460 & 66 & OR &  433 \\
        7 &  K &   103 & 22 & OUT & 431 & 37 & OUT & 453 & 52 & S &   606 & 67 & RST & 460 \\
        8 &  STL & 600 & 23 & OUT & 451 & 38 & K &   15 &  53 & LD &  455 & 68 & K &   5 \\
        9 &  OUT & 432 & 24 & K &   10 &  39 & LD &  453 & 54 & ANI & 460 & 69 & LD &  434 \\
        10 & OUT & 433 & 25 & LD &  451 & 40 & S &   606 & 55 & S &   610 & 70 & OUT & 460 \\
        11 & LD &  400 & 26 & S &   603 & 41 & STL & 606 & 56 & STL & 610 & 71 & END & \\
        12 & OR &  401 & 27 & STL & 603 & 42 & OUT & 454 & 57 & OUT & 433 & & & \\
        13 & S &   601 & 28 & OUT & 430 & 43 & K &   0.5 & 58 & OUT & 456 & & & \\
        14 & S &   604 & 29 & OUT & 452 & 44 & LD &  454 & 59 & K &   5 & & & \\
      \end{tabular}
    \end{table}
\section{課題}
\section{感想}
\section*{参考文献}
  \begin{enumerate}
    \item 令和元年度電子制御工学実験 $\cdot$ 3年後期テキスト
  \end{enumerate}
\end{document}
