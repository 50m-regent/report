\documentclass[titlepage]{jsarticle}
\usepackage[dvipdfmx]{graphicx}

\title{4年前期体育レポート課題\\副題: 高専生は運動不足か?}
\author{Ec 4年\\32番\\平田 蓮}
\date{}

\begin{document}
\maketitle

\section{副題についてのレポート}
    \subsection{これまでの生活}
        まずはじめに、今までの生活記録からこれまでの生活について振り返える。
        ここでは5/27から6/8までの生活を中心にみていく。

        まず、就寝時間と起床時間についてだが、基本的に早寝早起きを心がけることができていた。
        睡眠時間については、6から7時間が多く、規則正しい生活につながっていると考えられる。

        次に食事についてだが、基本的に昼食を摂っていないことが多い。これは高専に入学してから今まで続いている習慣である。
        昼食を抜くことの良し悪しについては数多くの議論が交わされている。

        最後に運動についてだが、犬を飼っていることもあり、基本的に毎日散歩もしくはランニングを行えている。
        また、たまに友達とバドミントンをしたりした。

        ちょうど一年ほど前からオンラインランニングがマイブームでずっと続けているが、
        コロナの影響で、オンライン〇〇という言葉がよく使われるようになり、オンラインランニング仲間の間でもこのことが話題となった。

    \subsection{活動量の基準METs}
        活動量の指標として、第10回の授業で学んだMETsというものがある。
        これは各運動についてその運動の強度を示す値である。

        例として、1時間のウォーキングの活動量を求めてみる。ウォーキングの運動強度は3.0メッツとされている。
        これに活動時間である1時間を掛けることで、活動量の3.0メッツ時が求まる。

        ここで、厚生労働省の示す活動量の指標をみてみると、
        日常生活では23メッツ時/週、スポーツ等として4メッツ時/週と示されている。
        これはどちらも3メッツ以上の運動についてである。

    \subsection{ある高専生の活動量}
        これらをもとに、ある一日の私の生活の活動量が基準に達しているかを調べてみる。
        
        まず、一日の行動を列挙してみる。

        \begin{itemize}
            \item ランニング(30分、5km)
            \item 登下校
            \item 買い物
            \item 犬の散歩(30分、2km)
        \end{itemize}

        上のうち、登下校を1時間のウォーキング、買い物を30分のウォーキングと捉える。
        この運動を日常生活と運動に分けて活動量に換算してみる。

        まず日常活動について、
        $3.5 メッツ \times 1 時間 + 3.3 メッツ \times 0.5 時間 + 3.0 メッツ \times 0.5 時間 = 6.65メッツ時$

        次に運動について、
        $9.8 メッツ \times 0.5 時間 = 4.9 メッツ時$

        一方活動量の基準値を一日あたりにすると、それぞれ約3.29メッツ時、約0.571メッツ時となる。
        これを私の活動量と比較すると、どちらも基準値を十分に満たしていることがわかる。

\section{授業の感想}
    今年は遠隔授業ということで、任意の実技授業があった。5人の寮生が参加している中、唯一の通生として前半の実技授業に参加した。

    いつもの授業とは違い、少人数で楽しみつつ様々な種目を行うことができてとても楽しかった。
    また、課題で出た小運動など、友達と家にいるときに一緒に実践をできた。なかなかない機会であったと思う。

    また、家にいながらも適切な量の運動を行うことができていたと思うので、今後もこの調子で続けていきたい。

\begin{thebibliography}{99}
    \bibitem{history} 一日二食生活の意外なメリット、効果まとめ Triffle \\
        https://trifle-gentoku.com/2020/01/22/post-390/
        (閲覧日: 2020/9/1)
    \bibitem{history} 昼食を抜くとなぜ良くないの? MEDIVA \\
            https://mediva.co.jp/hsd/blog/2017/06/28/774/
        (閲覧日: 2020/9/1)
    \bibitem{history} 最適な睡眠時間って何時間? 睡眠リズムラボ \\
        https://www.otsuka.co.jp/suimin/column02.html
        (閲覧日: 2020/9/1)
    \bibitem{} メッツ/METs e-ヘルスネット \\
        https://www.e-healthnet.mhlw.go.jp/information/dictionary/exercise/ys-004.html \\
        (閲覧日: 2020/9/1)
    \bibitem{} 栄養・代謝研究部 国立健康・栄養研究所 \\
        https://www.nibiohn.go.jp/eiken/programs/program\_kiso.html
        (閲覧日: 2020/9/1)
    \bibitem{} 健康づくりのための身体活動基準 健康長寿ネット \\
        https://www.tyojyu.or.jp/net/kenkou-tyoju/kenkou-zoushin/kenkou-kijun.html \\
        (閲覧日: 2020/9/1)
\end{thebibliography}
\end{document}