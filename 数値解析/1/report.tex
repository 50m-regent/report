\documentclass{jsarticle}
\usepackage[dvipdfmx]{graphicx}
\usepackage{bm}
\usepackage{amsmath}
\usepackage{amssymb}
\usepackage{amsfonts}
\usepackage{comment}
\usepackage{listings}
\lstset{
    basicstyle={\ttfamily},
    identifierstyle={\small},
    commentstyle={\smallitshape},
    keywordstyle={\small\bfseries},
    ndkeywordstyle={\small},
    stringstyle={\small\ttfamily},
    frame={tb},
    breaklines=true,
    columns=[l]{fullflexible},
    numbers=left,
    xrightmargin=0zw,
    xleftmargin=3zw,
    numberstyle={\scriptsize},
    stepnumber=1,
    numbersep=1zw,
    lineskip=-0.5ex,
    keepspaces=true,
    language=c
}
\renewcommand{\lstlistingname}{リスト}
\makeatletter
\newcommand{\figcaption}[1]{\def\@captype{figure}\caption{#1}}
\newcommand{\tblcaption}[1]{\def\@captype{table}\caption{#1}}
\makeatother

\title{数値解析レポートNo.1}
\author{32番 平田 蓮}
\date{}

\begin{document}
\maketitle
    \section{任意の行列の2行を交換する}
        作成した行列の任意の2行を交換する関数\verb|swap_row()|をリスト\ref{src:swap}に示す.

        \begin{lstlisting}[caption=swap.c, label=src:swap]
void swap_row(double **a, int row, int col, int x, int y) {
    if (x < 0 || x >= row || y < 0 || y >= row) {
        puts("Defective Input (swap_row())");
        return;
    }

    double *tmp = a[x];
    a[x] = a[y];
    a[y] = tmp;
}\end{lstlisting}

        引数の\verb|a|にはNAbasic.cの\verb|csvRead()|から得られる
        行列を渡す.
        \verb|row|, \verb|col|, \verb|x|, \verb|y|はそれぞれ
        行列の行数, 列数, 交換する行である.
        正方行列を掛け合わせることで行を交換することも可能だが,
        それだと正方行列以外の行列に対応できないため,
        リスト\ref{src:swap}のように
        2行のポインタを入れ替えることで実装した.

        この関数を使って課題4の
        $
            A = \left(
                \begin{array}{ccc}
                  2 & -1 & 5 \\
                  -4 & 2 & 1 \\
                  8 & 2 & -1
                \end{array}
            \right)
        $
        の1, 3行目を交換した際の出力を下に示す.

        \begin{verbatim}
Before swapping
2.000000 -1.000000 5.000000 
-4.000000 2.000000 1.000000 
8.000000 2.000000 -1.000000 

After swapping
8.000000 2.000000 -1.000000 
-4.000000 2.000000 1.000000 
2.000000 -1.000000 5.000000 
        \end{verbatim}

        正しく交換できていることがわかる.

    \section{行列積を利用して内積を求める}
        2つのベクトル
        $
            \left(
                \begin{array}{c}
                x_1 \\
                x_2 \\
                x_3
                \end{array}
            \right),
            \left(
                \begin{array}{c}
                y_1 \\
                y_2 \\
                y_3
                \end{array}
            \right)
        $
        の内積は
        $
            \left(
                \begin{array}{c}
                    x_1 \\
                    x_2 \\
                    x_3
                \end{array}
            \right) \cdot
            \left(
                \begin{array}{c}
                    y_1 \\
                    y_2 \\
                    y_3
                \end{array}
            \right) =
            x_1y_1+x_2y_2+x_3y_3
        $と表される.
        これは次の行列積
        $
            \left(
                \begin{array}{ccc}
                    x_1 & x_2 & x_3
                \end{array}
            \right)
            \left(
                \begin{array}{c}
                    y_1 \\
                    y_2 \\
                    y_3
                \end{array}
            \right)
        $
        で表すことができる.

        行列積を計算する関数\verb|matrix_p()|をリスト\ref{src:prod}に示す.

        \begin{lstlisting}[caption=prod.c, label=src:prod]
double **matrix_p(
    double **x, int xrow, int xcol,
    double **y, int yrow, int ycol
) {
    int i, j, k;
    double **ret = allocMatrix(xrow, ycol);

    if (xrow != ycol || xcol != yrow) {
        puts("Defective data! (matrix_p())");
        return ret;
    }

    for (i = 0; i < xrow; i++) {
        for (j = 0; j < ycol; j++) {
            for (k = 0; k < xcol; k++) {
                ret[i][j] += x[i][k] * y[k][j];
            }
        }
    }

    return ret;
}\end{lstlisting}
        
        引数は計算をする2つの行列とそれぞれの行数と列数である.
        
        2つのベクトルを読み込む際に,
        行ベクトルと列ベクトルとして読み込まなければ計算ができない.

        この関数を使用して課題2の2つのベクトルの内積を求めた結果を下に示す.

        \begin{verbatim}
Vector1
1.000000 5.000000 8.000000 

Vector2
3.000000 
7.000000 
-2.000000 

Inner product: 22.000000
        \end{verbatim}

        正しく計算できていることがわかる.

    \section{2つのベクトルの成す角を求める}
        2つのベクトル$\bm{v_1}, \bm{v_2}$
        の内積は2つの成す角を$\theta$としたとき,
        $\bm{v_1} \cdot \bm{v_2} = v_1v_2\cos{\theta}$
        と定義される.
        よって, 2つの成す角は,
        $\displaystyle \theta = \arccos{\frac{\bm{v_1} \cdot \bm{v_2}}{v_1 v_2}}$
        として求めることができる.
        これを求める関数\verb|calc_angle()|をリスト\ref{src:calc}に示す.

        \begin{lstlisting}[caption=calc\_angle.c, label=src:calc]
double calc_angle(
    double **x, int xrow, int xcol, 
    double **y, int yrow, int ycol
) {
    if (xrow != 1 || ycol != 1 || xcol != yrow) {
        puts("Defective data! (calc_angle())");
        return -1;
    }

    double p = matrix_p(x, y, xrow, xcol, yrow, ycol)[0][0];
    return acosf(
        p /
        norm(matrix2colVector(x, xrow, xcol), xrow * xcol) /
        norm(matrix2colVector(y, yrow, ycol), yrow * ycol)
    );
}\end{lstlisting}

        引数は\verb|matrix_p()|と同様である.

        これを使用して課題2の2つのベクトルの成す角を求めた結果を下に示す.

        \begin{verbatim}
Vector1
1.000000 5.000000 8.000000 

Vector2
3.000000 
7.000000 
-2.000000 

Inner product: 22.000000

Angle:  1.271850 [rad]
       72.871616 [°]
        \end{verbatim}

        わかりやすいように孤度法と度数法2通りで出力するようにした.

    \section{感想}
        課題の内容は比較的簡単に熟すことができた.
        2本のベクトルの成す角について,
        グラフにプロットをして実際の角度を確かめようとしたが,
        3次元プロットではわかりづらかったため,
        ここには載せなかった.
\end{document}