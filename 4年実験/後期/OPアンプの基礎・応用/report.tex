\documentclass[titlepage]{jsarticle}
\usepackage[dvipdfmx]{graphicx}
\usepackage{amsmath}
\usepackage{amssymb}
\usepackage{amsfonts}
\usepackage{comment}
\usepackage{h31ec-exp}
\usepackage{listings}
\lstset{
    basicstyle={\ttfamily},
    identifierstyle={\small},
    commentstyle={\smallitshape},
    keywordstyle={\small\bfseries},
    ndkeywordstyle={\small},
    stringstyle={\small\ttfamily},
    frame={tb},
    breaklines=true,
    columns=[l]{fullflexible},
    numbers=left,
    xrightmargin=0zw,
    xleftmargin=3zw,
    numberstyle={\scriptsize},
    stepnumber=1,
    numbersep=1zw,
    lineskip=-0.5ex,
    keepspaces=true,
    language=c
}
\renewcommand{\lstlistingname}{リスト}
\makeatletter
\newcommand{\figcaption}[1]{\def\@captype{figure}\caption{#1}}
\newcommand{\tblcaption}[1]{\def\@captype{table}\caption{#1}}
\makeatother

\title{OPアンプの基礎・応用}
\grade{4年32番}
\author{平田 蓮}
\team{}
\date{2020年12月11日}
\expdate{2020年11月19日, 11月26日, 12月3日, 12月10日}
\coauthor{}

\begin{document}
\maketitle
\section{目的}
    本実験では, アナログ回路の素子としてよく用いられるOPアンプを用いたアナログ
    回路の中から, 増幅回路, 演算回路取り上げ, その特性を理解することを目的とする.

    まず, 基本回路として反転増幅回路の特性を理解する.
    次に, 反転増幅回路を基本とした演算増幅回路として,
    微分回路, 積分回路を取り上げる. さらに応用回路として,
    信号に含まれる特定の周波数成分を取り出すフィルタ回路を設計し,
    その周波数特性を測定する.

\section{OPアンプの基本動作}
    OPアンプについて, 詳しくは実験テキスト\cite{text}
    に載っているためここでは代表的な特徴のみ示す.

    \begin{itemize}
        \item 閉ループ利得が非常に大きい(理想的には無限大)
        \item 入力インピーダンスが非常に大きい(理想的には無限大)
        \item 出力インピーダンスが低い(理想的には$0 \ \Omega$)
    \end{itemize}

    OPアンプをアナログ回路に利用する際は負帰還回路として用いることが多い.
    今回扱う回路も全て負帰還の回路である.

\section{反転増幅回路}
    図\ref{fig:inv-amp}に反転増幅回路を示す.

    この回路について, 閉ループ利得$G \ [倍]$, $A \ [\rm{dB}]$を求める.
    OPアンプの性質より, 二つの入力電圧を$v_s$とするとこれらは等しくなるので,
    二つの抵抗を流れる電流$i_i, i_f$について以下の式が成り立つ.

    \begin{equation*}
        i_i = \frac{v_i - v_s}{R_i} = i_f = \frac{v_s - v_o}{R_f}
    \end{equation*}

    ここで, 理想的なOPアンプでの閉ループ利得を$A_0 \rightarrow \infty$とすると,
    $\displaystyle v_s = \frac{v_o}{A_0} = 0$となるので,

    \begin{equation}
        \frac{v_i}{R_i} = -\frac{v_o}{R_f}
    \end{equation}

    となる. よって, 閉ループ利得は,

    \begin{eqnarray}
        G &=& \left|\frac{v_o}{v_i}\right| = \frac{R_f}{R_i} \\
        A &=& 20 \log_{10}G
    \end{eqnarray}

    と表せる.

    \subsection{周波数特性の測定}

    \subsection{課題1}
        \paragraph{}

        \paragraph{}

        \paragraph{}

\section{微分回路}
    
    \subsection{基本微分回路}

        \subsubsection{入出力波形の観察}

        \subsubsection{周波数特性の測定}

    \subsection{実用微分回路}

        \subsubsection{周波数特性の測定}

    \subsection{課題2}

\section{積分回路}

    \subsection{基本積分回路}

        \subsubsection{入出力波形の観察}

        \subsubsection{周波数特性の測定}

    \subsection{実用積分回路}

        \subsection{周波数特性の測定}

    \subsection{課題3}

\section{フィルタ回路}

    \subsection{周波数特性の測定}
    
\begin{thebibliography}{99}
    \bibitem{text} OPアンプの基礎・応用, 電子制御工学実験$\cdot$ 4年後期テキスト, 2020
\end{thebibliography}
\end{document}