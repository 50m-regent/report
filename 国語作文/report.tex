\documentclass[]{tarticle}
\pagestyle{empty}

\begin{document}
「空き時間は一つのことに集中するか、もしくは様々な活動にバランスよく費やすか、どちらが良いと考えますか。」

試験の記述問題などでこの問を目にしたことが何度かある。
僕はいつもこう答える。

「たくさんのことを体験したいのでいろいろなことに浅く広く挑戦する。」

これは、何か一つを突き詰めている人が悪いと言っているわけではない。
多くの経験をしたいというのは、以前の経験が人生の中で役に立つ瞬間あると考えているからだ。
実際に、はじめは興味本位でやってみたが今は趣味や特技になっていることもある。

就職という場面での「役に立つ」を考えてみる。
ほとんどの人間にとって労働は人生と切り離せない義務である。
必然的に誰もが「やるなら好きなことを仕事にしたい」と考えるだろう。

では、{\bf 好きなことを仕事にする}とはどういうことだろうか。

読書の好きな友人がいる。彼女は大好きな本に囲まれて仕事ができると言い、
高校生になってすぐに本屋でアルバイトを始めた。
しかしこれは彼女が実際にやりたかった仕事ではなかった。
なぜなら彼女が好きなのは読書で、本の陳列や客の相手ではなかったからだ。

もう一つ例を挙げてみる。
僕はウェブ開発の仕事をしたことがある。
その仕事はとても楽しく、いい経験ができたと思っている。
しかし、あの仕事を生涯通して続けるかというとそれは違った。
なぜなら僕が好きなのは自分の思い描いたものを創ることで、
企業の一員として指示通りに作業をすることではなかったからだ。
{\bf 好きなこと}と一概にいってもいざそれを仕事にしたときに自分が満足できるとは限らないのである。

このように、{\bf 好きなことを仕事にする}のはよく目にする文言にしては多様化していて、ハードルが高い。
このハードルを将来超えるために僕は自分の中の{\bf 好きなこと}を増やしたいと考えている。
たくさんのことに挑戦して培った経験をいつか仕事という形で活かせれば、それはとても楽しいことではないだろうか。
\end{document}