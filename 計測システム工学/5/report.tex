\documentclass{jsarticle}
\usepackage[dvipdfmx]{graphicx}
\usepackage{bm}
\usepackage{amsmath}
\usepackage{amssymb}
\usepackage{amsfonts}
\usepackage{comment}
\usepackage{listings}
\usepackage{cases}
\usepackage{siunitx}
\usepackage[hyphens]{url}
\lstset{
    basicstyle={\ttfamily},
    identifierstyle={\small},
    commentstyle={\smallitshape},
    keywordstyle={\small\bfseries},
    ndkeywordstyle={\small},
    stringstyle={\small\ttfamily},
    frame={tb},
    breaklines=true,
    columns=[l]{fullflexible},
    numbers=left,
    xrightmargin=0zw,
    xleftmargin=3zw,
    numberstyle={\scriptsize},
    stepnumber=1,
    numbersep=1zw,
    lineskip=-0.5ex,
    keepspaces=true,
    language=c
}
\renewcommand{\lstlistingname}{リスト}
\makeatletter
\newcommand{\figcaption}[1]{\def\@captype{figure}\caption{#1}}
\newcommand{\tblcaption}[1]{\def\@captype{table}\caption{#1}}
\makeatother

\title{計測システム工学 第五回課題}
\author{Ec5 24番 平田 蓮}
\date{}

\begin{document}
\maketitle
\section{零位法、偏位法の同種の用語に"置換法"、"補償法"、"差動法"がある。どのような測定法であるかを調べよ。可能であれば例を挙げよ。}
    \paragraph{置換法} 測定量と基準値を置き換えて2回の測定結果から測定量を知る方法 \\
        例: 天秤で分銅の重さを測り、次に計測したいものを乗せることでそれの重さを知る。
    \paragraph{補償法} 計測量と既知量との差を測る \\
        例: 自分の体重を測り、次にペットを抱っこして体重を測り、その差からペットの体重を知る。
    \paragraph{差動法} 作用する二つの量の差が対象 \\
        例: 差動変圧器を用いた測定
\section{検出器の内部インピーダンスを考慮して出力電圧の補正を行えば、伝送器は必要ないと言えるか。言えないとすればその理由を答えよ。}
    伝送器の重要な役割として、インピーダンス変換とレベル変換がある。
    出力電圧を補正することでレベル変換は賄えるが、
    入出力のインピーダンスの変換は行えないため、必要である。
\end{document}