\documentclass{jsarticle}
\usepackage[dvipdfmx]{graphicx}
\usepackage{bm}
\usepackage{amsmath}
\usepackage{amssymb}
\usepackage{amsfonts}
\usepackage{comment}
\usepackage{listings}
\usepackage{cases}
\lstset{
    basicstyle={\ttfamily},
    identifierstyle={\small},
    commentstyle={\smallitshape},
    keywordstyle={\small\bfseries},
    ndkeywordstyle={\small},
    stringstyle={\small\ttfamily},
    frame={tb},
    breaklines=true,
    columns=[l]{fullflexible},
    numbers=left,
    xrightmargin=0zw,
    xleftmargin=3zw,
    numberstyle={\scriptsize},
    stepnumber=1,
    numbersep=1zw,
    lineskip=-0.5ex,
    keepspaces=true,
    language=c
}
\renewcommand{\lstlistingname}{リスト}
\makeatletter
\newcommand{\figcaption}[1]{\def\@captype{figure}\caption{#1}}
\newcommand{\tblcaption}[1]{\def\@captype{table}\caption{#1}}
\makeatother

\title{計測システム工学 第二回課題}
\author{Ec5 24番 平田 蓮}
\date{}

\begin{document}
\maketitle
\subsection*{以下のように、正規母集団から無作為に抽出した20個分のデータが得られているとき、以下の問いに答えよ。}
    \begin{table}[h]
        \centering
        \begin{tabular}{cccccccccc}
            3.1 & 4.0 & 2.9 & 3.2 & 5.1 & 3.6 & 4.1 & 3.3 & 4.0 & 4.0 \\
            4.4 & 4.3 & 2.8 & 3.0 & 4.2 & 4.8 & 4.4 & 5.0 & 5.3 & 4.5 
        \end{tabular}
    \end{table}

    \subsubsection*{1) 母分散$\sigma ^2=1.0$が既知である場合、母平均$m$の95\%信頼区間を求めよ。}
        母標準偏差は$\sigma = 1.0$、試料平均は$\bar{x} = 4.0$

        母平均の分布は正規分布である。
        95\%の確率に対応する$\displaystyle\alpha = \frac{1 - 0.95}{2} = 0.025$について、
        標準正規分布のパーセント点を表から調べると、
        $z(\alpha) = a(0.025) = 1.9600$だから、
        母平均$m$の90\%信頼区間は
        \begin{equation*}
            4.0 - 1.9600\frac{1.0}{\sqrt{20}} \leq m \leq 4.0 + 1.9600\frac{1.0}{\sqrt{20}}
            \leftrightarrow 3.56 \leq m \leq 4.44
        \end{equation*}

    \subsubsection*{2) 母分散(母標準偏差)が未知の場合、母平均$m$の95\%信頼区間を求めよ。}
        試料不偏分散は、$\displaystyle s_n^2=\frac{(3.1-4.0)^2+(4.0-4.0)^2+\cdots +(4.5-4.0)^2}{19}=0.58$

        $\therefore$試料標準偏差は$s_n=0.76$

        母平均の分布はt分布である。
        $\displaystyle\alpha = \frac{1 - 0.95}{2} = 0.025, n=20$について、
        t分布のパーセント点を表から調べると、
        $t_{n-1}(\alpha)=t_19(0.025)=2.093$だから、
        母平均$m$の95\%信頼区間は
        \begin{equation*}
            4.0 - 2.093\frac{0.76}{\sqrt{20}} \leq m \leq 4.0 + 2.093\frac{0.76}{\sqrt{20}}
            \leftrightarrow 3.64 \leq m \leq 4.36
        \end{equation*}

    \subsubsection*{3) 母平均$m = 3.9$が既知である場合、母分散$\sigma ^2$の90\%信頼区間を求めよ。}
        $nS_0^2=(3.1-3.9)^2+(4.0-3.9)^2+\cdots +(4.5-3.9)^2=11.2$

        $\displaystyle\alpha = \frac{1 - 0.9}{2} = 0.05, n=20$
        に対応する$\chi ^2$分布のパーセント点を調べると、

        $\chi_n^2(\alpha)=\chi_{20}^2(0.05)=31.41, \chi_n^2(1-\alpha)=\chi_{20}^2(0.95)=10.85$だから、
        母分散の90\%信頼区間は
        \begin{equation*}
            \frac{11.2}{31.41} \leq \sigma^2 \leq \frac{11.2}{10.85}
            \leftrightarrow 0.357 \leq m \leq 1.03
        \end{equation*}

    \subsubsection*{4) 母平均が未知の場合、母分散$\sigma ^2$の90\%信頼区間を求めよ。}
        母平均が未知なので、試料平均$\bar{x}=4.0$を用いて

        $nS^2=(3.1-4.0)^2+(4.0-4.0)^2+\cdots +(4.5-4.0)^2=11.0$

        $\displaystyle\alpha = \frac{1 - 0.9}{2} = 0.05, n-1=19$
        に対応する$\chi ^2$分布のパーセント点を調べると、

        $\chi_{n-1}^2(\alpha)=\chi_{19}^2(0.05)=30.14, \chi_{n-1}^2(1-\alpha)=\chi_{19}^2(0.95)=10.12$だから、
        母分散の90\%信頼区間は
        \begin{equation*}
            \frac{11.2}{30.14} \leq \sigma^2 \leq \frac{11.2}{10.12}
            \leftrightarrow 0.372 \leq m \leq 1.11
        \end{equation*}

\end{document}