\documentclass{jsarticle}
\usepackage[dvipdfmx]{graphicx}
\usepackage{bm}
\usepackage{amsmath}
\usepackage{amssymb}
\usepackage{amsfonts}
\usepackage{comment}
\usepackage{listings}
\usepackage{cases}
\lstset{
    basicstyle={\ttfamily},
    identifierstyle={\small},
    commentstyle={\smallitshape},
    keywordstyle={\small\bfseries},
    ndkeywordstyle={\small},
    stringstyle={\small\ttfamily},
    frame={tb},
    breaklines=true,
    columns=[l]{fullflexible},
    numbers=left,
    xrightmargin=0zw,
    xleftmargin=3zw,
    numberstyle={\scriptsize},
    stepnumber=1,
    numbersep=1zw,
    lineskip=-0.5ex,
    keepspaces=true,
    language=c
}
\renewcommand{\lstlistingname}{リスト}
\makeatletter
\newcommand{\figcaption}[1]{\def\@captype{figure}\caption{#1}}
\newcommand{\tblcaption}[1]{\def\@captype{table}\caption{#1}}
\makeatother

\title{計測システム工学 第三回課題}
\author{Ec5 24番 平田 蓮}
\date{}

\begin{document}
\maketitle
\subsection*{1. 角度の測定結果が$\theta + \Delta\theta [\mathrm{rad}]$であるとき、$\cos\theta$の誤差の式を求めよ。}
    \begin{equation*}
        \Delta\cos\theta = \frac{\partial}{\partial\theta}\cos\theta\Delta\theta = -\sin\theta\Delta\theta
    \end{equation*}

\subsection*{2. 測定値が$x + \Delta x$であるとき、$q = e^x$の誤差の式を求めよ。}
    \begin{equation*}
        \Delta q = \frac{\partial q}{\partial x}\Delta x = e^x\Delta x
    \end{equation*}

\subsection*{3. $q=x^2$の総合誤差率は、$\displaystyle\frac{\Delta q}{q} = 2\frac{\Delta x}{x}$と考えるべきか、$q = x \cdot x$であることから$\displaystyle\frac{\Delta q}{q} = \sqrt{\left(\frac{\Delta x}{x}\right)^2 + \left(\frac{\Delta x}{x}\right)^2}=\sqrt{2}\frac{\Delta x}{x}$と考えるべきか。}
    $x$は単一の量であるので、$x^2 = x \cdot x$とは考えずに計算を行う。

    \begin{equation*}
        \therefore \displaystyle\frac{\Delta q}{q} = 2\frac{\Delta x}{x}
    \end{equation*}

\subsection*{4. 単振り子の周期から重力加速度を求めたい。測定の結果、振り子の長さが$l\pm\varepsilon_l$、周期が$T\pm\varepsilon_T$であったとき、重力加速度$g$を求めよ。}
    単振り子の周期は$\displaystyle T = 2\pi\sqrt{\frac{l}{g}}$であるので、
    $\displaystyle g = \frac{4\pi^2l}{T^2}$

    よって、$g$の総合誤差を$\varepsilon_g$とすると、
    $\displaystyle g = \frac{4\pi^2l}{T^2}\pm\varepsilon_g$

    ここで、$\displaystyle\varepsilon_g = \frac{4\pi^2l}{T^2}\sqrt{1^2\left(\frac{\varepsilon_l}{l}\right)^2+(-2)^2\left(\frac{\varepsilon_T}{T}\right)^2}=\frac{4\pi^2l}{T^2}\sqrt{\left(\frac{\varepsilon_l}{l}\right)^2+4\left(\frac{\varepsilon_T}{T}\right)^2}$
    であるので、

    \begin{equation*}
        g = \frac{4\pi^2l}{T^2}\left(1\pm\sqrt{\left(\frac{\varepsilon_l}{l}\right)^2+4\left(\frac{\varepsilon_T}{T}\right)^2}\right)
    \end{equation*}

\end{document}